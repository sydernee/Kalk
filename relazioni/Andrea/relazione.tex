\documentclass[10pt,a4paper]{article}

\usepackage[english,italian]{babel}
\usepackage[utf8]{inputenc}
\usepackage{hyperref} 
\hypersetup{
    bookmarks=true,         % show bookmarks bar?
    % unicode=false,          % non-Latin characters in Acrobat’s bookmarks
    % pdftoolbar=true,        % show Acrobat’s toolbar?
    pdfmenubar=true,        % show Acrobat’s menu?
    % pdffitwindow=false,     % window fit to page when opened
    pdfstartview={FitH},    % fits the width of the page to the window
    pdftitle={Kalk - Timoty Granziero},    % title
    pdfauthor={Timoty Granziero},     % author
    % pdfsubject={},   % subject of the document 
    % pdfcreator={Creator},   % creator of the document
    % pdfproducer={Producer}, % producer of the document
    % pdfkeywords={keyword1, key2, key3}, % list of keywords
    % pdfnewwindow=true,      % links in new PDF window
    colorlinks=true,       % false: boxed links; true: colored links
    linkcolor=black,          % color of internal links (change box color with linkbordercolor)
    % citecolor=green,        % color of links to bibliography
    % filecolor=magenta,      % color of file links
    urlcolor=cyan           % color of external links
}

\usepackage[margin=1.3in]{geometry}

\usepackage{graphicx} %img/media
\usepackage{xcolor} %colors

\usepackage{enumitem} %lists
\usepackage[perpage]{footmisc} %footnote, starting at 1 every page

\newcommand{\authorName}{Timoty Granziero}

%liste
\renewcommand{\labelitemi}{$\circ$}
\renewcommand{\labelitemii}{$\cdot$}
\renewcommand{\labelitemiii}{$\diamond$}
\renewcommand{\labelitemiv}{$\ast$}

\author{\authorName}
\date{\today}
\title{Relazione del Progetto di Programmazione ad Ogetti}

\usepackage[htt]{hyphenat} %allow \texttt to line break

\usepackage{fancyhdr} %header e footer

%header/footer setup
\fancyhf{}
\fancyhead[R]{\small\scshape\nouppercase{\leftmark}}
\fancyhead[L]{\small\scshape\nouppercase{\rightmark}}
%\fancyhead[L,R]{\small\thepage}
\lhead{\nouppercase{\rightmark}}
\rhead{\nouppercase{\leftmark}}
\cfoot{\thepage}
%\lfoot{\authorName}

\pagestyle{fancy}

\begin{document}
    \begin{titlepage} 
	\centering
	\includegraphics[width=0.50\textwidth]{img/logo.pdf}\par\vspace{1cm} %logo 
	
	{\LARGE\bfseries Relazione Progetto di \par Programmazione ad Oggetti \par}
	\vspace{1cm}
	
	{\Large\bfseries a.a. 2017/2018 \par}
	
	\vspace{1cm} 

	Autore: \par
    {\bfseries \authorName \par} 
    Matricola 1026100 \par 
	
	\vspace{1cm}
    
    \vfill
	
	% Bottom of the page
	{\large \today\par}
	
\end{titlepage}

    \clearpage
    
    \tableofcontents
    \clearpage

\section{Introduzione}
Per il progetto didattico di programmazione ad oggetti, è stato sviluppato il software \textit{Kalk}:
una calcolatrice che supporta l'uso di diversi tipi di dato, a scelta dello studente. Il progetto
è stato sviluppato in coppia con Timoty Granziero. \par

Sono stati individuati i seguenti tipi di calcolo:
    \begin{itemize}
        \item \texttt{Matrix}, da cui derivano \texttt{SquareMatrix}
        e \texttt{SparseMatrix};
        \item \texttt{Network}, che usa la classe \texttt{User}.
    \end{itemize}

\noindent{}Il progetto, come da specifica, è stato sviluppato in C++ e Java.

\section{Compilazione ed esecuzione}
\subsection{C\texttt{++}} 
Per compilare il progetto in \texttt{C++}, è stato fornito il file \texttt{kalk.pro}
all'interno della cartella \textbf{cpp}. Posizionarsi all'interno di quella cartella ed eseguire il comando
\texttt{qmake} per generare il Makefile, poi \texttt{make} per compilare i sorgenti. A questo punto, sar\`a presente
un eseguibile denominato \texttt{kalk}; è sufficiente eseguirlo con il comando \texttt{./kalk} (mentre si è posizionati
all'interno della cartella cpp) per lanciare l'applicativo.

\subsection{Java} 
Per compilare il progetto in \texttt{Java}, compilare i sorgenti \texttt{java} dalla directory \textbf{java}
del progetto con il comando 
\begin{center}
    \texttt{javac kalkException/*.java dataTypes/*.java} \par
\end{center} 
Dopodichè eseguire lo script della classe \texttt{Use} invocando il comando (sempre dalla cartella java)
\begin{center}
    \texttt{java dataTypes.Use}
\end{center}

   
\section{Ambiente di sviluppo}
Per lo sviluppo del software, ho scelto di utilizzare sia la macchina virtuale fornita dai tecnici di
laboratorio che i PC del dipartimento, e quindi il \textbf{Sistema Operativo} Ubuntu 16.04, \textbf{compilatore}
GCC 5.4.0, QT 5.7 e Java 8.

\subsection{Editor}
Per lo sviluppo del software utilizzato, nel mio caso, è stato il semplice editor \textbf{gedit}, in coppia con 
il plugin \textbf{wakatime}, il quale permette di misurare esattamente il tempo trascorso utilizzando l'editor:
tale software è disponibile anche per altri editor e IDE, purtroppo non per QtCreator.

\subsection{Sistema di versionamento}
Per la condivisione dei file versionabili all'interno del gruppo, è stato scelto l'uso del sistema di versionamento
Git e la piattaforma di Github.

\section{Tempo di sviluppo} 
Al fine di completare lo sviluppo del software, si sono rese necessarie, da parte mia, approssimativamente di 55 ore:
30 per la parte del modello e le restanti per l'interfaccia.
 

\subsection{Suddivisione del lavoro progettuale}
La suddivisione del lavoro è stata la seguente:
\begin{itemize}
    \item \textbf{Andrea}:
    \begin{itemize}
        \item Contributo allo sviluppo e al debugging della gerarchia \texttt{Matrix} in \texttt{C++};
        \item Gerarchia \texttt{Network} in \texttt{C++} e \texttt{Java};
        \item sviluppo della GUI di \texttt{Network}, usando le QT.
    \end{itemize}
    \item \textbf{Timoty}:
    \begin{itemize}
        \item Sviluppo della gerarchia di \texttt{Matrix} in \texttt{C++} e \texttt{Java};
        \item GUI di \texttt{Matrix}, usando le QT.
    \end{itemize}
\end{itemize}


\section{Tipi di dato sviluppati e scelte progettuali}
Si è deciso, in comune accordo, di adottare il design pattern MVC: esso separa il modello,
dal controller e dalla vista: la struttura delle cartelle rispecchia tale scelta.

\subsection{Modello}
La classe datatype è la classe base astratta a capo della gerarchia della parte logica del modello:
ogni classe che è stata intesa come tipo di dato deriva da essa.

Lo sviluppo del modello è avvenuto in maniera completamente separata dalla vista e dal controller,
quindi non subisce da essi nessuna influenza.

\subsubsection{Matrix}
Modella una matrice algebrica di dimensione qualsiasi: le operazioni consentite sono quelle classiche
di una matrice, tra cui: il calcolo della somma, differenza, il prodotto per scalare e tra matrici, trasposta.

\subsubsection{SquareMatrix}
La classe modella una matrice quadrata, derivando da Matrix.
In aggiunta a Matrix, permette il calcolo del determinante, il minore della matrice e di determinare se la matrice è triangolare superiore,
triangolare inferiore, diagonale o simmetrica.
Oltre a questo, permete di generare una SquareMatrix di zeri o identità, base alla dimensione stabilita.

\subsubsection{SparseMatrix}
SparseMatrix è un'altra classe derivata da Matrix. Una matrice sparsa è una normale matrice con buona parte delle 
celle con valore nullo (0). Aggiunge i campi sparsity, al fine di calcolare la sparsità della matrice
in maniera più efficiente.
Permette inoltre di calcolare la densità della matrice e di ottenere le righe e le colonne non a zero della matrice

\subsection{Polimorfismo in Matrix}


    Si è deciso, nella definizione degli operatori binari, di privilegiare la commutatività, quindi dovendo dichiarare
    esternamente gli operatori, piuttosto che optare per il polimorfismo.    

    Per il resto, gran parte delle funzioni sono dichiarate come virtual, anche dove non era necessario effettuare l'override.

\subsubsection{Network}
Network modella un social network, rappresentandone gli utenti e i connessioni tra di loro, alias follow.
Nello sviluppo della classe è stato sfruttato il framework Qt, in particolare per la costruzione delle tuple
follower e followed e l'uso dei puntatori smart.

 Le funzioni fornite dalla classe sono:
 \begin{itemize}
     \item \texttt{size()} restituisce il numero di utenti nella rete;
     \item \texttt{addUser()} aggiunge un utente alla rete (se non già presente, altrimenti lancia un'eccezione);
     \item \texttt{isUserOfTheNetwork()} restituisce un booleano che è \texttt{true} se e solo se l'utente è presente
     nella rete;
     \item \texttt{removeUser()} rimuove un utente con un determinato \texttt{username};
     \item \texttt{addFollower()} aggiunge una coppia (\texttt{QPair}) di puntatori a utenti a \texttt{followed} (il primo segue
     il secondo);
     \item \texttt{removeFollower()} rimuove una coppia da \texttt{following};
     \item \texttt{isFollowerOf()} verifica se il primo \texttt{User} segue il secondo;
     \item \texttt{getFollower()} restituisce un contenitore (\texttt{QSet}) con i puntatori a utenti seguiti da un certo \texttt{User};
     \item \texttt{getFollowed()} restituisce un contenitore (\texttt{QSet}) con i puntatori a utenti che seguono un certo \texttt{User}.
 \end{itemize}

Infine sono presenti una serie di operazioni insiemistiche tra due \texttt{Network}: 
\begin{itemize}
    \item \texttt{getUnion()} restituisce un \texttt{QSet} in cui sono presenti i puntatori a tutti gli utenti
    in comune e non tra le due reti;
    \item \texttt{getIntersection()} restituisce un \texttt{QSet} in cui sono presenti i puntatori a tutti gli 
    utenti in comune tra le due reti;
    \item \texttt{getRelativeComplement()} restituisce un \texttt{QSet} che è il complemento relativo tra gli utenti 
    delle due reti;
    \item \texttt{getSymmetricDifference()} restituisce un \texttt{QSet} che è la differenza simmetrica tra gli utenti
    delle due reti.
\end{itemize}


\section{Java}
Le classi in java rappresentano essenzialmente il modello di kalk: lo sviluppo è stato, tutto sommato, abbastanza agevole.

\section{Osservazioni e difficoltà}
Le due principali difficoltà nel progetto sono state l'individuazione di un numero sufficiente di un tipo sensato di 
tipi di dato e lo studio della libreria Qt: data la sua vastità, è facile perdersi nella sua documentazione senza
arrivare a una conclusione, in particolare per quanto riguarda il design pattern MV.

\end{document}
